\chapter{Introduction}

In a modern society the amount of information is far behind what one can remember or even process. If we understand that, we realize how important it is to be able to delegate the thinking amongs group. The decision making in groups and teams is a topic covered by several papers. [citation required] Supposing we have limited capacity of memory, we have to distribute the knowledge amongst people around us and communicate with each other so as to gather facts which we currently need to make the decision. 

Our decisions are either consciously or subliminaly based on our needs or drives - former term might be rather connected with human behavior, latter term is used for plausible agents. As in microeconomics we can use an utility as a measure of relative satisfaction \cite{Varian:micro} and see how one manages fulfilling their needs. While attaining the goals we use a knowledge which we store in our memory and which we update regularly. With infinite memory we wouldn’t have any problems to store all information and use it when required; however, we don’t have such memory - our memory is limited. 

What I mean by saying “not to have enough space in our memory” is one is not able to remember everything. Certain pieces of information are fading away as time goes or as one is learning new facts. I want to observer if and how an intensive communication can substitute insufficient memory space with the condition of constant level of utility.

Is it obvious that adding the ability of communication improves the chance

I want to demonstrate that there is a relation between amount of communication and needed space in memory (...)

This thesis is consist of N parts. First, I will introduce the topic of agent and possible memory implementations based on concrete examples. (...)